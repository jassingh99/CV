%%%%%%%%%%%%%%%%%%%%%%%%%%%%%%%%%%%%%%%%%%%%%%%%%%%%%%%%%%%%%%%%%%%%%%%%%%%%%%%%
% Medium Length Graduate Curriculum Vitae
% LaTeX Template
% Version 1.2 (3/28/15)
%
% This template has been downloaded from:
% http://www.LaTeXTemplates.com
%
% Original author:
% Rensselaer Polytechnic Institute 
% (http://www.rpi.edu/dept/arc/training/latex/resumes/)
%
% Modified by:
% Daniel L Marks <xleafr@gmail.com> 3/28/2015
%
% Important note:
% This template requires the res.cls file to be in the same directory as the
% .tex file. The res.cls file provides the resume style used for structuring the
% document.
%
%%%%%%%%%%%%%%%%%%%%%%%%%%%%%%%%%%%%%%%%%%%%%%%%%%%%%%%%%%%%%%%%%%%%%%%%%%%%%%%%

%-------------------------------------------------------------------------------
%	PACKAGES AND OTHER DOCUMENT CONFIGURATIONS
%-------------------------------------------------------------------------------

%%%%%%%%%%%%%%%%%%%%%%%%%%%%%%%%%%%%%%%%%%%%%%%%%%%%%%%%%%%%%%%%%%%%%%%%%%%%%%%%
% You can have multiple style options the legal options ones are:
%
%   centered:	the name and address are centered at the top of the page 
%				(default)
%
%   line:		the name is the left with a horizontal line then the address to
%				the right
%
%   overlapped:	the section titles overlap the body text (default)
%
%   margin:		the section titles are to the left of the body text
%		
%   11pt:		use 11 point fonts instead of 10 point fonts
%
%   12pt:		use 12 point fonts instead of 10 point fonts
%
%%%%%%%%%%%%%%%%%%%%%%%%%%%%%%%%%%%%%%%%%%%%%%%%%%%%%%%%%%%%%%%%%%%%%%%%%%%%%%%%

\documentclass[margin]{res}  

% Default font is the helvetica postscript font
\usepackage{helvet, xcolor}
\usepackage[utf8]{inputenc}

\usepackage{hyperref}
\hypersetup{
    colorlinks=true,
    linkcolor=blue,
    filecolor=magenta,      
    urlcolor=cyan,
}

% Increase text height
\textheight=700pt

\begin{document}

\name{Jas Singh}

\address{\href{mailto:jas@math.ucla.edu}{jas@math.ucla.edu}}
\address{\url{https://www.math.ucla.edu/~jas/}}

% Uncomment to add a third address
%\address{Address 3 line 1\\Address 3 line 2\\Address 3 line 3}
%-------------------------------------------------------------------------------

\begin{resume}

%-------------------------------------------------------------------------------
%	EDUCATION SECTION
%-------------------------------------------------------------------------------
\section{EDUCATION}
\textbf{University of California Los Angeles} \hfill \textbf{September 2016 - March 2020}\\
{\sl Bachelor of Science}, Mathematics\\

\textbf{University of California Los Angeles} \hfill \textbf{September 2021 - }\\
{\sl Ph.D}, Mathematics\\
\section{ACADEMICS}

As an undergraduate at UCLA I took rigorous and challenging courses in mathematics, including many honors upper division and graduate courses. I also took and passed the spring 2019 Ph.D qualifying exam in Algebra as an undergraduate.

\textbf{197 Courses}\\
In spring 2018, I worked with Professor Moschovakis and a small group of interested students in a Math 197 course, wherein we studied topics in set theory such as ordinals and descriptive set theory. We met twice weekly to present solutions to problems we had solved.

In fall 2019, I read algebraic number theory with Professor Totaro at UCLA as a Math 197 course. This was a continuation of readings I did with Professor Totaro the summer before. I read parts of Janusz's \textit{Algebraic Number Fields} and completed exercises in the book. Professor Totaro and I met weekly to discuss these readings and problems.

\section{EXTRA-CURRICULARS}

\textbf{Directed Reading Program}\\
Throughout the academic year of 2018-2019 I was involved in the UCLA mathematics department's directed reading program (DRP). In this, I was paired with a graduate student in math as a mentor who would help me engage in an independent reading project. I read through sections of \textit{Algebraic Theories: A Categorical Introduction to General Algebra} by Adámek, Rosický, and Vitale as well as \textit{Higher Categories and Homotopical Algebra} by Cisinski. I gave a ten minute presentation to fellow undergraduate students in the DRP about my reading in algebraic theories at the end of the fall quarter of that year.

\textbf{Reading Courses}\\
Over summer 2018 I engaged in a guided reading with Professor Elman at UCLA. I read the sections and did exercises on real closed fields and multilinear algebra in his book \textit{Lectures on Algebra}, which can be found on \href{https://www.math.ucla.edu/~rse/}{his website}. I met Professor Elman weekly to discuss these readings.

I did a guided reading with Professor Totaro over the summer of 2019. I read parts of Professor Sharifi's algebraic number theory notes, which can be found on \href{http://math.ucla.edu/~sharifi/lecnotes.html}{his website}. Professor Sharifi provided me with relevant exercises which I worked through. I met Professor Totaro weekly to discuss these readings and problems.

In winter 2020 I read parts of Huybrechts' \textit{Complex Geometry} with Professor Totaro. We met roughly every week to discuss these readings. I got through the first four chapters, doing various exercises along the way.

\section{TECHNICAL SKILLS}

I have basic proficiency in \LaTeX{} and have typeset numerous lecture notes and homework sets.\\

\section{FELLOWSHIPS AND AWARDS}
I received the 2020 Sherwood Prize from UCLA's mathematics department for exceptional performance as an undergraduate.
\end{resume}
\end{document}
